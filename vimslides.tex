\documentclass{beamer}

\usepackage{beamerthemesplit}

\title{Intermediate Vim}
\date{\today}

\begin{document}

\frame{\titlepage}

% Slide 
\section{Introduction}
\subsection{Prerequisites}
\frame 
{
    \frametitle{Prerequisites}
    \textbf{Saving -} :wq\\
    \textbf{Editing -} :e\\
    \textbf{Moving Around -} hjkl, wb, C-u, C-d, zz etc... \\
    \textbf{Regular Expressions -} s///g, filenames
}

% Slide 
\subsection{Overview of topics}
\frame
{
    \frametitle{Overview of topics}

    \begin{enumerate}
        \item Set and Map
        \item Marks and Registers
        \item Macros
        \item Laying out a Work Session
        \begin{itemize}
            \item Buffers
            \item Windows
            \item Tabs
        \end{itemize}
        \item Programming
        \begin{itemize}
            \item Compilation (Make)
            \item Runtime (GDB)
        \end{itemize}

    \end{enumerate}
}

% Slide 
\section{Essentials}
\subsection{Set}
\frame
{
    \frametitle{Set Sets Variables}

    \begin{itemize}
        \item Variables are used by the editor
        \item Variables are used in Vimscript
        \item Keycodes are contained in variables
    \end{itemize}
}
\subsection{Map}
\frame
{
    \frametitle{Map binds keys}

    \begin{itemize}
        \item WYTIWYG
        \item Remember the $<$CR$>$
    \end{itemize}
}
\subsection{Marks}
\frame
{
    \frametitle{Marks get you around quickly}

    \begin{itemize}
        \item First mark your target with m
        \item Get there quickly with $`$
        \item :marks
    \end{itemize}
}
\subsection{Registers}
\frame
{
    \frametitle{Registers are clipboards}

    \begin{itemize}
        \item yank it into a named register
        \item put it with ``
        \item edit it with ``
        \item put it again
    \end{itemize}
}

% Slide
\subsection{Macros}
\frame
{
    \frametitle{Surprise! Macros are just registers}

    \begin{itemize}
        \item Recording Macros
        \item Playing Macros Back (Thanks Chad!)
        \item Editing Macros: The `` registers
    \end{itemize}
}

% Slide
\section{Laying out a Work Session}
\frame
{
    \frametitle{Different views on the same document set}

    \begin{itemize}
        \item Buffers
        \item Windows
        \item Tabs
    \end{itemize}
}

% Slide
\subsection{Buffers}
\frame
{
    \frametitle{Buffers}

    \begin{itemize}
        \item Buffers are the files you have open
        \item :buffers
        \item :b close
        \item :bufdo
    \end{itemize}
}

% Slide
\subsection{Windows}
\frame
{
    \frametitle{Windows}

    \begin{itemize}
        \item Windows offer a view
        \item :sp and :vs
        \item :b autocompletion
        \item :windo
    \end{itemize}
}

% Slide
\subsection{Tabs}
\frame
{
    \frametitle{Tabs}

    \begin{itemize}
        \item An additional view layer
        \item :tabnew
        \item Remap to standards
        \item :tabdo
    \end{itemize}
}

% Slide 
\section{Programming}
\frame
{
    \frametitle{Never leave vim}

    \begin{itemize}
        \item Compilation and Runtime Debugging
        \item Make is well integrated
        \item GDB, not so much
    \end{itemize}
}

% Slide 
\subsection{Compilation (Make)}
\frame
{
    \frametitle{Compilation (Make)}

    \begin{itemize}
        \item :make
        \item :cwin and the quickfix window (thanks Finch!)
        \item :ccl
    \end{itemize}
}

% Slide 
\subsection{Runtime (GDB)}
\frame
{
    \frametitle{Runtime (GDB)}

    \begin{itemize}
        \item gdbvim
        \item cgdb
        \item clewn/vimgdb/pyclewn
    \end{itemize}
}

% Slide 
\section{Outro}
\frame
{
    \frametitle{Outro}

    \begin{itemize}
        \item vim++
        \item http://vim.wikia.com/
    \end{itemize}
}
\end{document}

